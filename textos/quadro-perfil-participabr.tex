\begin{quadro}[htb]
    \IBGEtab{
        \renewcommand{\arraystretch}{1.4}
        \ABNTEXfontereduzida
    \caption{\label{qua:participabr-perfil}Participa.br - Entrevistados} %todo
  }{%
    %\resizebox{1.5\textwidth}{!}{
      \tiny
      \begin{tabular}{|P{2.8cm}|p{1.6cm}|p{6cm}|p{4cm}|}
        \hline
          \headerCenterCell{Nome} &
          \headerCenterCell{Data da Entrevista} &
          \headerCenterCell{Caracterização} &
          \headerCenterCell{Motivo para entrevista} \\
        \hline \hline
        \mbox{\textbf{Ricardo Augusto}} \mbox{\textbf{Poppi Martins}} &
          21/06/2014&
          Ricardo Augusto Poppi Martins, 36 anos, natural de São Paulo (SP), é Coordenador Geral de Novas Mídias e Outras Linguagens de Participação (Coordenação de Novas Mídias), inserido no Departamento de Participação Social da Secretaria Nacional de Articulação Social que pertence à Secretaria Geral da Presidência da República (SGPR). Ele trabalha no governo federal desde 2010, como servidor comissionado, sendo que nunca havia trabalhado antes em órgãos públicos. &
          O Participa.br, portal federal de participação social, está sob sua coordenação direta. \\
        \hline
          \mbox{\textbf{Paulo Roberto}} \mbox{\textbf{Miranda Meirelles}} &
          19/06/2014&
          Paulo Roberto Miranda Meirelles, 31 anos, natural de Natal (RN), é professor da Universidade de Brasília (UnB) no curso de Engenharia de Software. Atualmente contribui com desenvolvimento do Participa.br pois conta com uma equipe na UnB que colabora para o Noosfero, que é a plataforma livre base do portal federal em questão e também consultor PNUD da SGPR com o objetivo é desenvolver um biblioteca digital sobre participação social para o Participa.br. &
          De dezembro de 2013 a abril de 2014 esteve envolvido com o Participa.br alinhando o desenvolvimento da plataforma Noosfero entre as instituições UnB, USP, SERPRO e SGPR, evitando retrabalho e garantindo convergência. Além disso, a biblioteca digital a respeito da participação social também.\\
        \hline
        \mbox{\textbf{Daniela Soares}} \mbox{\textbf{Feitosa}} &
          16/06/2014&
          Daniela Soares Feitosa, 30 anos, natural Salvador (BA) é consultora PNUD junto à Secretaria Geral da Presidência da República (SGPR) com o objetivo de pensar em formas e funcionalidades que fomentem a participação social por meio da plataforma, incorporando-as no código do Participa.br.&
          Sua visão como desenvolvedora e sua experiência com Noosfero, base do Participa.br. \\
        \hline
          \textbf{Graziele Machado} &
          19/06/2014&
          Graziele Machado, 30 anos, natural Brasília (DF) é consultora PNUD junto à Secretaria Geral da Presidência da República (SGPR) com o objetivo de fomentar participação social e mobilização por meio de novas mídias e produção de conteúdo jornalístico, inserido numa lógica 2.0. Já havia trabalhado em 2008/2009 com o Ministério da Cultura e em 2011 com o Ministério do Desenvolvimento, Indústria e Comércio Exterior, como assessora de comunicação.&
          Ela é responsável por produção de conteúdo e com foco dos processos comunicativos em si.\\
        \hline
          \textbf{Mariel Zasso} &
          12/06/2014&
        Mariel Zasso, 31 anos, natural de Nova Palma (RS) é consultora PNUD junto à Secretaria Geral da Presidência da República (SGPR) com a missão de produzir um concurso de aplicativos (Hackathon) das OSCs, não trabalhando diretamente com o Participa.br.&
        Sua experiência e atuação dentro das comunidades do Participa.br, como uma usuária e técnica “customizadora” da rede social. Ela foi consultora do Ministério da Justiça para auxiliar na gestão da comunidade COMIGRAR, criada por conta da 1ª Conferência Nacional sobre Migrações e Refúgio. Esta experiência merece certo destaque tanto pela oportunidade de entrevistar uma stakeholder protagonista do processo, quanto por ser a maior comunidade (não endógena) do Participa.br, com 208 participantes. \\
        \hline
      \end{tabular}
      \normalsize
    %}
  }{%
    \fonte{compilação própria}
  }
\end{quadro}
