% ---
% Capitulo Considerações Finais
% ---
\chapter{Algumas Considerações e Próximos Passos}\label{chap:considfinais}
% META: 5p.

A divisão do trabalho de acordo com o gênero implica diferentes padrões de atividades e, assim, diferentes padrões de viagens. As mulheres desempenham diferentes papéis no mercado de trabalho e também na família e, embora tenham conseguido diminuir as desigualdades no mercado de trabalho ao longo do tempo, não vêm obtendo o mesmo êxito em relação ao trabalho doméstico. A existência de filhos também influencia marcadamente o padrão de atividades da família e principalmente da mulher, vista como a principal responsável por seus cuidados. A sobrecarga do trabalho doméstico ser deixada para o lado feminino da família implica menos disponibilidade de tempo e a configuração de uma cidade menor para as mulheres. Outrossim, resulta numa maior pressão por utilização de modos de transportes que ofereçam mais velocidade e flexibilidade de itinerário, ou seja, um incentivo ao uso do carro.
Porém, mesmo o automóvel sendo a forma de maior atratividade entre os modos de transporte, ainda assim as mulheres o utilizam menos em relação aos homens.

Buscar identificar essas diferenças e entender as estratégias utilizadas pelas mulheres em seu acesso às atividades é objetivo desta pesquisa. O acesso feminino em geral mais restrito ao automóvel particular não as impede de realizar mais atividades durante o dia e delineia um padrão de deslocamentos que colabora para um desenvolvimento mais sustentável. Seria então desejável às mulheres buscarem o padrão masculino insustentável ou o inverso? Ao compreender melhor as estratégias e limitações que prevalecem tanto no universo feminino como masculino, este trabalho pode contribuir para a formulação de políticas públicas que estimulem comportamentos menos dependentes do uso do automóvel.

Certas conclusões já podem ser depreendidas da revisão de literatura, como a acessibilidade ser um dos pontos chave no entendimento do padrões de deslocamentos. Garantir que haja acesso às oportunidades (de trabalho, estudo, lazer, compras, saúde) de forma mais distribuída no espaço urbano torna possível a utilização de modos não motorizados, mais sustentáveis. Dada a existência de oportunidades mais próximas à residência, é preciso também que o ambiente construído seja convidativo a realizar as viagens a pé ou de bicicleta, ou seja, as pessoas de qualquer gênero devem sentir-se seguras e acolhidas pela cidade que as cerca. 
Dentro no grupo dos modos motorizados, tornar o transporte público mais acessível a todos(as) é uma condição para atingir padrões de deslocamentos mais sustentáveis. Entre as iniciativas possíveis nesse sentido é preciso considerar a questão da promoção da equidade. Nas decisões relativas à infra-estrutura do sistema de transporte quais modos terão prioridade nos espaço de circulação viário, como promover a capilarização da rede e como tornar financeiramente acessível as tarifas a todas pessoas.

Na conexão entre gênero, mobilidade e sustentabilidade, \citeauthoronline{HANSON2010} (\citeyear{HANSON2010}, p.7) aponta que é preciso mudar a agenda da pesquisa na direção em que se sintetizem três dimensões: localidade, abordagens quantitativas e qualitativas, e modos de pensar transversais a gênero e mobilidade. O presente trabalho faz esforço no sentido de pensar transversalmente gênero, mobilidade e sustentabilidade considerando o período o período de 1977 a 2007 da RMSP.
Com esse intuito, a partir da revisão da literatura, foi identificada uma sorte de variáveis que podem auxiliar a compreender o comportamento das pessoas (ou grupos de pessoas), como por exemplo: número de viagens, distância percorrida, duração das viagens, modo utilizado, motivo declarado (na origem e no destino),  encadeamento de viagens. Além disso, também é possível olhar as distâncias residência-trabalho e residência-escola, assim como a duração das atividades e não apenas das viagens. Esse é o desafio metodológico a ser enfrentado na próxima fase do trabalho: procurar entender, ainda que parcialmente, o fenômeno complexo que é o comportamento humano, partindo dos dados já disponíveis apresentados.
A seguir, são elencadas algumas possibilidades de caminhos, que podem ser seguidos individual ou conjuntamente:

(i) Identificação de relações entre variáveis relevantes a partir do uso de técnicas de árvore de decisão \cite{MITCHELL1997,  PITOMBO2013}. Dentro desta técnica há vários algoritmos possíveis \cite{BREIMAN1984}
%http://www.statsoft.com/Textbook/Classification-and-Regression-Trees 
como o CART \cite{DENISON1998,FABRICIUS2000} e o CHAID, que definem grupos homogêneos de uma população tendo em vista uma variável resposta \cite{MAGIDSON1994, STRAMBI1998}.

(ii) Utilização de modelo logit multinomial em que os anos sejam representados por variáveis \emph{dummies}, assim como a variável sexo, em conjunto com outras variáveis relevantes, determinadas a partir da revisão de literatura e/ou de alguma técnica de segmentação aplicada em etapa anterior.

(iii) Organização de um modelo de análise longitudinal, que consiga captar alterações comportamentais ao longo do tempo para gerações (coortes) \cite{DARGAY2000,STRAMBI2000}, como por exemplo através de um pseudo-painel \cite{DARGAY2002,BRESSON2004,WARUNSIRI2010,NETTO2014a}%
\footnote{Um painel é caracterizado pela composição de diversas \emph{corss-sections} dos mesmos indivíduos ao longo do tempo \cite{DEATON1985,VERBEEK1992,WOOLDRIDGE2002}.
No caso das Pesquisas OD, não há o acompanhamento dos mesmos indivíduos ao longo do tempo. Assim, pode-se ao invés de olhar para o mesmo indivíduo, olhar uma mesma ``célula de análise'' \cite{DEATON1985,NETTO2014}, dando origem a um pseudo painel. 
No caso dos paineis genuínos, existem dois tipos de variações que precisam ser controladas: aquelas eu ocorrem entre os indivíduos (\emph{between}) e aquelas que ocorrem para um mesmo indivíduo, ao longo do tempo (\emph{within}) \cite{FAVERO2013}. No pseudo-painel talvez ainda seja necessário controlar econometricamente outro efeito, advindo do fato de que não são os mesmos indivíduos a serem observados no tempo.}
%RELATAR AS DUAS POSSÍVEIS FORMAS DE ANÁLISE APONTADAS POR DARGAY...
% Dados em painel: \cite{DEATON1985,VERBEEK1992,VERBEEK1992,WOOLDRIDGE2002,FAVERO2013}
% Pseudo painel: \cite{MENG2014,NETTO2014a}
% Em transportes (painel): \cite{HANSON1985,MOFFITT1993,WIZEMAN2001,BEST2005,THOGERSEN2006, TRB2006,GLAESER2008,DIANA2010,SENER2011,SCHEINER2013,COMPTOM2014,DICIOMMO2014,PERCHOUX2014, RASOULI2014}

(iv) Estudo da evolução da interação entre variáveis explicativas fazendo análises de semelhanças a partir de dados da diversas \emph{cross-sections}. Por exemplo, pode-se olhar o encadeamento de viagens \cite{GOULIAS1990} buscando agrupar ``famílias'' de viagens \cite{DALMASO2009} ou mesmo de ``prismas espaço-tempo'' que possuam características semelhantes. A partir desses agrupamentos classificatórios pode-se observar quais são as características sociodemográficas individuais comuns às ``famílias'' de cadeias de viagens ou prismas espaço-tempo, e quais são discriminantes.

(v) Para investigar motivos específicos que os dados secundários não são suficientes é possível também realizar um \emph{survey online}. Embora o \emph{survey online} possa introduzir vieses em relação à amostragem, ele pode ter um alcance de maior, tanto em número de pessoas, quanto em abrangência geográfica. Neste caminho, seria possível esmiuçar melhor as viagens a pé ou coletar dados relacionados a raça/etnia, pois segundo \apudonline[p.8]{PRATT1994}{HANSON2010}:

\begin{citacao}
Os processos que definem o gênero são sempre declinados  por outras dimensões de diferenças percebidas (idade, etnia, por exemplo) e desenvolvem-se nas práticas cotidianas em voga, incluindo aquelas relacionadas à mobilidade.
\end{citacao}

(vi) Uma pesquisa qualitativa também pode ser uma saída interessante, pois tem maior foco no da interpretação do que na quantificação, ênfase na subjetividade, maior flexibilidade na condução do processo de pesquisa e necessidade de preocupação com o contexto \cite{CASELL1994}. Aqui, poderiam ser feitas entrevistas buscando entender melhor a dinâmica familiar sob aspectos como: distribuição dos recursos, divisão de tarefas, poder de decisão sobre local de residência e trabalho, etc. As entrevistas podem ocorrer fora do seu domicílio e, talvez nessa situação, as pessoas sintam-se mais à vontade para falar do seu cotidiano - numa pesquisa domiciliar muitas vezes se entrevista um indivíduo da família na presença de outro. Contudo, a quantidade de pessoas e locais alcançados são limitados pela disponibilidade de recurso (tempo e dinheiro) dos(as) pesquisadores(as). 

Por fim, uma abordagem não exclui outra necessariamente, podendo integrar-se e complementar-se num mesmo projeto \cite{MINAYO1996}. Ainda é espaço para avançar a escolha do método, que sabe-se ser função tanto da natureza do problema e do nível de aprofundamento desejado \cite{DIEHL2004}, como dos recursos materiais e imateriais disponíveis.
Neste contexto, foi traçada uma sequência de macro-atividades prevista para a próxima fase desta pesquisa, a ser observada no Quadro \ref{qua:cronograma} a seguir.

\begin{quadro}[htb]
    \IBGEtab{
        \renewcommand{\arraystretch}{1.5}
        \ABNTEXfontereduzida
        \caption[Cronograma]{\label{qua:cronograma}Cronograma das atividades a serem desenvolvidas no mestrado em 2015}
	}{%
        \begin{tabular}{|p{3.0cm}|P{0.72cm}|P{0.72cm}|P{0.74cm}|P{0.72cm}|P{0.72cm}|P{0.72cm}|P{0.72cm}|P{0.72cm}|P{0.72cm}|P{0.72cm}|}
           \hline
		       \headerCenterCell{Atividade} & 
		       \headerCenterCell{Jan} & 
		       \headerCenterCell{Fev} & 
		       \headerCenterCell{Mar} &
   		       \headerCenterCell{Abr} & 
   		       \headerCenterCell{Mai} & 
   		       \headerCenterCell{Jun} & 
   		       \headerCenterCell{Jul} & 
   		       \headerCenterCell{Ago} & 
   		       \headerCenterCell{Set} & 
   		       \headerCenterCell{Out}\\ 
		    \hline\hline
		        Avanço na revisão da literatura de acordo com sugestões da banca de qualificação&
		        X &
		        X &
		        X &
		        X &
		        X &
		        &
		        &
		        &
		        &
		        \\
		    \hline
		    	Preparação do banco de dados reunindo OD-77, OD-87, OD-97 e OD-07&
		        X &
		        &
		        &
		        &
		        &
		        &
		        &
		        &
		        &
		        \\
		    \hline
		    	Avanço da pesquisa quantitativa e/ou qualitativa&
		        X &
		        X &
		        X &
		        &
		        &
		        &
		        &
		        &
		        &
		        \\
		    \hline
		    	Tratamento de dados e análise de resultados&
		        &
		        &
		        &
		        X &
		        X &
		        X &
		        X &
		        &
		        &
		        \\
		    \hline
		    	Redação da dissertação&
		        X &
		        X &
		        X &
		        X &
		        X &
		        X &
		        X &
		        X &
		        &
		        \\
   		    \hline
		    	Revisão do texto da dissertação&
		        &
		        &
		        &
		        &
		        &
		        &
		        &
		        &
		        X &
		        \\
		    \hline
		    	Defesa da dissertação de mestrado diante banca&
		        &
		        &
		        &
		        &
		        &
		        &
		        &
		        &
		        &
		        X\\
		    \hline
		\end{tabular}
	}{%
		\fonte{Elaboração própria}
    }
\end{quadro}

%Logo, tendo em vista a elaboração de políticas de transporte público que o torne mais atrativo para as mulheres, é preciso primeiramente garantir que o transporte público seja acessível. Isto significa tanto não haver barreiras econômicas (tarifárias), já que mulheres têm rendimento médio inferior a homens, quanto haver capilaridade suficiente da rede, para que haja a percepção de que existe transporte público ``à disposição'', assim como se tem com o carro. Em segundo lugar é preciso que o(s) modo(s) seja(m) adequado(s) à atividade que será desempenhada. Para mulheres, se considerarmos as viagens motivo ``manutenção'' da casa, acompanhar crianças à escola ou ao médico e ir às compras são atividades relevantes. No Reino Unido, Hamilton e Jenkins (2000) apontam para a falta de adequação da infraestrutura de transportes às necessidades socialmente atribuídas às mulheres. No Brasil, o diagnóstico não difere à exceção de poucos municípios como Curitiba: é impossível uma mulher utilizar um ônibus com um carrinho de bebê ou carrinho de compras. 

%Por fim, constatando a influência da divisão do trabalho sobre o padrão de mobilidade de acordo com o gênero, podem ser tomadas medidas de incentivo ao uso do transporte público de forma a considerar a diversidade de necessidades que levam as pessoas a se locomoverem nas cidades (GTZ, 2007). São exemplos de medidas que podem ser adotadas: incremento da infra-estrutura do sistema de transporte existente, capilarização da rede, mudança na metodologia de pesquisa origem-destino para que se passe a considerar viagens a pé inferiores a 500m ou trechos a pé na cadeia de viagens. Estas mudanças podem aumentar a eficácia e eficiência do sistema de transporte urbano, já que este será mais atrativo - não apenas para as mulheres - e pode contribuir com a diminuição dos congestionamentos.
