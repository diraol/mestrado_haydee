% ---
% Capitulo Considerações Finais
% ---
\chapter{Algumas Considerações e Próximos Passos}\label{chap:considfinais}
% META: 5p.

\hl{-> terminar até 26 de novembro}

%Entre os vários modos de transporte, o automóvel é a forma geralmente de maior atratividade. Mas, apesar de seus apelos, um grupo particular de pessoas chama a atenção em relação ao uso diferenciado que fazem do automóvel: as mulheres. Historicamente, estas usam menos o automóvel em relação aos homens. Esta pesquisa de mestrado busca identificar e entender as estratégias utilizadas pelas mulheres em seu acesso em geral mais restrito ao automóvel particular e como este fato afeta seus padrões de atividades e viagens, com o objetivo de formular políticas públicas que estimulem comportamentos similares, menos dependentes  do uso do automóvel.

%A divisão do trabalho de acordo com o gênero implica diferentes padrões de atividades e, assim, diferentes padrões de viagens. As mulheres desempenham diferentes papéis no mercado de trabalho e também na família, elas têm conseguido diminuir as desigualdades no mercado de trabalho ao longo do tempo mas não vêm obtendo o mesmo êxito em relação ao trabalho doméstico. A existência de filhos também influencia marcadamente o padrão de atividades da família e principalmente da mulher, principal responsável por seus cuidados. A sobrecarga do trabalho doméstico ser deixada para o lado feminino da família implica menos disponibilidade de tempo para as mulheres e resulta numa maior pressão por utilização de modos de transportes que ofereçam mais velocidade e flexibilidade de itinerário, ou seja, um incentivo ao uso do carro. Consequentemente, as mulheres que possuem filhos são as que apresentam a maior restrição de mobilidade.

%Logo, tendo em vista a elaboração de políticas de transporte público que o torne mais atrativo para as mulheres, é preciso primeiramente garantir que o transporte público seja acessível. Isto significa tanto não haver barreiras econômicas (tarifárias), já que mulheres têm rendimento médio inferior a homens, quanto haver capilaridade suficiente da rede, para que haja a percepção de que existe transporte público ``à disposição'', assim como se tem com o carro. Em segundo lugar é preciso que o(s) modo(s) seja adequado à atividade que será desempenhada. Para mulheres, se considerarmos as viagens motivo ``manutenção'' da casa, acompanhar crianças à escola ou ao médico e ir às compras são atividades relevantes. No Reino Unido, Hamilton e Jenkins (2000) apontam para a falta de adequação da infraestrutura de transportes às necessidades socialmente atribuídas às mulheres. No Brasil, o diagnóstico não difere à exceção de poucos municípios como Curitiba: é impossível uma mulher utilizar um ônibus com um carrinho de bebê ou carrinho de compras. 

%Por fim, constatando a influência da divisão do trabalho sobre o padrão de mobilidade de acordo com o gênero, podem ser tomadas medidas de incentivo ao uso do transporte público de forma a considerar a diversidade de necessidades que levam as pessoas a se locomoverem nas cidades (GTZ, 2007). São exemplos de medidas que podem ser adotadas: incremento da infra-estrutura do sistema de transporte existente, capilarização da rede, mudança na metodologia de pesquisa origem-destino para que se passe a considerar viagens a pé inferiores a 500m ou trechos a pé na cadeia de viagens. Estas mudanças podem aumentar a eficácia e eficiência do sistema de transporte urbano, já que este será mais atrativo - não apenas para as mulheres - e pode contribuir com a diminuição dos congestionamentos.

%Para \hl{Daus (apud FREITAG)} algumas cidades extra-europeias, mesmo com seu vertiginoso crescimento e urbanização, pode servir de exemplo de inovação e capacidade de adaptação.

Dados em painel: \cite{DEATON1985,VERBEEK1992,VERBEEK1992,WOOLDRIDGE2002,FAVERO2013}

pseudo painel: \cite{WARUNSIRI2010,MENG2014,NETTO2014,NETTO2014a}

RELATAR AS DUAS POSSÍVEIS FORMAS DE ANÁLISE APONTADAS POR DARGAY...

Em transportes (painel): \cite{HANSON1985,MOFFITT1993,WIZEMAN2001,LANZENDORF2005,THOGERSEN2006,TRB2006,GLAESER2008,DIANA2010,SENER2011,
SCHEINER2013,COMPTOM2014,DICIOMMO2014,PERCHOUX2014,RASOULI2014}

Em transportes (pseudo painel): \cite{DARGAY1999,DARGAY2002,BRESSON2004}

Além de fase da vida a pessoa encontra-se, seria interesante ainda averiguar fatores ligados ao ``estilo de vida'' conjugadamente à idade, como por exemplo, estado civil (casado(a), solteiro(a), viúvo(a), etc.), presença de criança na família e em qual faiza etária. Todavia, as bases das Pesquisas OD não fornecem tais dados.


Conforme pode ser observado no Quadro \ref{qua:cronograma} pretende-se ....

\begin{quadro}[htb]
    \IBGEtab{
        \renewcommand{\arraystretch}{1.5}
        \ABNTEXfontereduzida
        \caption[Cronograma]{\label{qua:cronograma}Cronograma das atividads a serem desenvolvidas no mestrado em 2015}
	}{%
        \begin{tabular}{|p{3.0cm}|P{0.72cm}|P{0.72cm}|P{0.74cm}|P{0.72cm}|P{0.72cm}|P{0.72cm}|P{0.72cm}|P{0.72cm}|P{0.72cm}|P{0.72cm}|}
           \hline
		       \headerCenterCell{Atividade} & 
		       \headerCenterCell{Jan} & 
		       \headerCenterCell{Fev} & 
		       \headerCenterCell{Mar} &
   		       \headerCenterCell{Abr} & 
   		       \headerCenterCell{Mai} & 
   		       \headerCenterCell{Jun} & 
   		       \headerCenterCell{Jul} & 
   		       \headerCenterCell{Ago} & 
   		       \headerCenterCell{Set} & 
   		       \headerCenterCell{Out}\\ 
		    \hline\hline
		        Avanço na revisão da literatura de acordo com sugestões da banca de qualificação&
		        X &
		        X &
		        X &
		        X &
		        X &
		        &
		        &
		        &
		        &
		        \\
		    \hline
		    	Preparação do banco de dados reunindo OD-77, OD-87, OD-97 e OD-07&
		        X &
		        &
		        &
		        &
		        &
		        &
		        &
		        &
		        &
		        \\
		    \hline
		    	Modelagem dos dados em pseudo-painel&
		        X &
		        X &
		        X &
		        &
		        &
		        &
		        &
		        &
		        &
		        \\
		    \hline
		    	Validação do modelo e análise de resultados&
		        &
		        &
		        &
		        X &
		        X &
		        X &
		        &
		        &
		        &
		        \\
		    \hline
		    	Redação da dissertação&
		        X &
		        X &
		        X &
		        X &
		        X &
		        X &
		        X &
		        X &
		        &
		        \\
   		    \hline
		    	Revisão do texto da dissertação&
		        &
		        &
		        &
		        &
		        &
		        &
		        &
		        &
		        X &
		        \\
		    \hline
		    	Defesa da dissertação de mestrado diante banca&
		        &
		        &
		        &
		        &
		        &
		        &
		        &
		        &
		        &
		        X\\
		    \hline
		\end{tabular}
	}{%
		\fonte{Elaboração própria}
    }
\end{quadro}

