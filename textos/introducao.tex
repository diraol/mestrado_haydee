% ----------------------------------------------------------
% Introdução (exemplo de capítulo sem numeração, mas presente no Sumário)
% ----------------------------------------------------------
\chapter{Introdução}
% ----------------------------------------------------------

A propagação de megacidades
\footnote{``Megacidade'' é um termo, cunhado em 1990 pela ONU, para designar cidades com mais de dez milhões de habitantes. Segundo dados da Divisão de População da ONU, em 2014 existem 33 megacidades no planeta e São Paulo é a sétima megacidade no \emph{ranking}. Fonte: \url{http://esa.un.org/unpd/ppp/} Acesso em 30 de outubro de 2014.
Segundo \citeauthoronline{FREITAG2007} (\citeyear{FREITAG2007}), São Paulo é também uma ``megalópole'', isto é, uma megacidade (município com mais de 10 milhões de habitantes) que sofreu um crescimento muito acelerado nas três ou quatro últimas décadas do século XX.}
%, a saber, em ordem descrescente de população: Tóquio (Japão), Delhi (Índia), Seul (Coreia do Sul), Shanghai (China), Mumbai (Índia), Cidade do México (México), São Paulo (Brasil), Beijing (China), Osaka (Japão), Nova Iorque (Estados Unidos), Jacarta (Indonésia), Manila (Filipinas), Karachi (Paquistão), Cairo (Egito), Los Angeles (Estados Unidos), Dhaka (Bangladesh), Moscou (Rússia), Buenos Aires (Argentina), Kolkata (Índia), Londres (Reino Unido), Istambul (Turquia), Bangkok (Tailândia), Rio de Janeiro (Brasil), Lagos (Nigéria), Tehran (Irã), Guangzhou (china), Kinshasa (República Democrática do Congo), Shenzhen (China), Lahore (Paquistão), Rhine-Ruhr (Alemanha), Paris (França), Tianjin (China) e Bangalore (Índia).
%
especialmente nos países em desenvolvimento, tem sido acompanhada de fenômenos como aumento da urbanização e da motorização. O Brasil já conta com duas megacidades: Rio de Janeiro e São Paulo, esta última é objeto deste estudo.
O que pouco tem acompanhado esses crescimentos têm sido as políticas de planejamento urbano e de transportes, de forma integrada, o que tem incorrido em problemas como congestionamentos \cite{KINGHAM2001,STENG2005,METZ2012}, piora das características ambientais \cite{TERTOOLEN1998,RICHARDSON2005,BANISTER2011}, e aprofundamento das desigualdades \cite{HODGE1995,AHMED2008,LEWIS2011}.
A inequidade não se dá apenas no acesso aos recursos e às oportunidades, mas também na distribuição dos espaços públicos \cite{ALVA1997} e, em específico, o espaço destinado à circulação nas cidades \cite{VASCONCELLOS2012}.

Estudos constatam \cite{VASCONCELLOS2001,RUEDA2007} que os automóveis particulares ocupam maior quantidade de espaço de circulação para transportar a mesma quantidade de pessoas do que outros modos de transporte (motorizados ou não).  
Tendo em vista esse cenário, recentemente, foi aprovada no Brasil a Política Nacional de Mobilidade Urbana \cite{PNMU} que implicitamente indica nos seus ``princípios, diretrizes e objetivos'' que não é mais aceitável a priorização do automóvel particular no meio urbano, já que deve promover a ``equidade no uso do espaço público de circulação, vias e logradouros''.

Neste cenário, justifica-se estudar alternativas que contribuam para uma mobilidade mais sustentável nos grandes centros urbanos, considerando que entre os vários modos de transporte, em geral, o automóvel particular é a forma de maior atratividade. Observa-se, entretanto, que apesar de seus apelos, um grupo particular de pessoas chama a atenção em relação ao uso diferenciado que fazem do automóvel: as mulheres.
Historicamente, estas usam menos o automóvel em relação aos homens \cite{FOX1983,HJORTHOL2000,POLK2003,BEST2005}.

Esta dissertação, portanto, aborda diretamente questões relativas a análises de comportamento da demanda por transportes, jogando luz sobre as questões de  gênero e suas articulações, principalmente com políticas de transporte e, em menor medida, com políticas de planejamento urbano. Entender melhor o comportamento da demanda pode colaborar na formulação de políticas de incentivo à troca de modos de transporte (de menos para mais sustentáveis) e à redução da necessidade de viajar.

\section{Objeto e Objetivos}
O objeto desta dissertação de mestrado é a relação entre o gênero e os deslocamentos de indivíduos. Assim, por objetivo geral tem-se investigar como o padrão de viagens de indivíduos é afetado pela categoria de análise gênero, no período de 1977 a 2007 na Região Metropolitana de São Paulo (RMSP). Toma-se por hipótese a ser explorada que pessoas com identidades de gênero masculina e feminina tem se deslocado de forma diferente no espaço.

Como objetivos específicos estão: %TODO usar ambiente de listagem?
(i) analisar padrões de viagens, por gênero, de cada \emph{cross-section}
\footnote{Dados em \emph{cross-section} são dados em seções transversais, ou seja, revelam características por meio das variáveis para um dado momento.};
(ii) analisar padrões de viagens, por modo, de cada \emph{cross-section};
(iii) comparar os padrões encontrados e verificar se existe alguma tendência ao longo do tempo; (iv) elaborar hipóteses sobre as motivações das mulheres e dos homens para realizar viagens, com base na teoria subjacente; (v) analisar criticamente se hipóteses elencadas são validadas pelos dados e resultados obtidos.



\section{Justificativa}

A questão de gênero nos transportes tem atraído atenção crescente da comunidade científica \cite{ROSENBLOOM1978,HANSON1985,ROSENBLOOM2006,UTENG2008,HANSON2010}.
Pesquisadores(as) começaram a examinar os padrões de mobilidade com o recorte de gênero considerando que há acesso desigual a recursos materiais e imateriais \cite{HOWE1982,HANSON1995,ELMHIRST2003,RAJU2005} que levam a diferenças nos padrões de atividades e de viagen\cite{FAGNANI1983,LAFFERTY1991,LAFFERTY1992,IBIPO1992,ROOT1999,SCHWANEN2002,SONG2003,
MCNUCKIN2005,CRANE2007,VASCONCELLOS2012}, em particular na escolha de modo e no uso do automóvel particular \cite{FOX1983,HJORTHOL2000,POLK2003,BEST2005}.

Com uma maior participação de mulheres no mercado de trabalho a diferença entre os rendimentos de homens e mulheres vem (lentamente) diminuindo, o que levaria a crer que o padrão de viagens das mulheres passasse a se assemelhar aos dos homens, pois teriam mais recursos financeiros para dispor de um automóvel. Pela flexibilidade de horário e de trajeto que o carro pode oferecer, seria esperado um aumento no uso do carro pelas mulheres devido à pressão exercida por uma dupla jornada (trabalho formal e doméstico) ainda mais se houver presença de criança na família. Entretanto, segundo estudo de  \citeauthoronline{BEST2005} (\citeyear{BEST2005}) na Alemanha, a presença de criança na família gera o seguinte efeito: a maternidade reduz a probabilidade de uso do carro por mulheres enquanto a paternidade a aumenta para homens.

Resultados como esse, aparentemente contraintuitivo, demonstram a necessidade de olhar mais atento e que considere complexidades sociais na análise do comportamento das demandas de transportes. Ademais, justifica a necessidade de compreender as estratégias adotadas pelas mulheres para cumprir com seu amplo conjunto de compromissos sem, muitas vezes e por motivos diversos, contar com o acesso ao automóvel, ou com acesso restrito a este modo de transporte.
Embora este tipo de análise seja cada vez mais frequente no cenário internacional, ainda é incipiente no Brasil. Assim, as lições extraídas deste estudo podem identificar oportunidades de formulação de políticas públicas que visem reduzir a dependência do uso automóvel particular, promovendo sistemas de transporte e estimulando comportamentos mais sustentáveis nas megacidades brasileiras.

\section{Método}

Em primeiro lugar, será conduzida uma revisão bibliográfica sobre gênero, sustentabilidade, transportes e suas intersecções conceituais, tendo como foco o estabelecimento de padrões de viagens e as escolhas modais.
A segunda etapa consistirá da caracterização da evolução do padrão de atividades e mobilidade das mulheres por meio de análise das Pesquisas Origem e Destino de 1977, 1987, 1997 e 2007 para a Região Metropolitana de São Paulo (RMSP).
A última etapa prevê uma pesquisa qualitativa com o propósito de buscar compreender as motivações comportamentais e estratégias de deslocamento de mulheres e homens na consecução de seu padrão de atividades.

\section{Estrutura do trabalho}

No primeiro capítulo é feita uma breve introdução ao tema, assim como são apresentados objetivos e justificativa deste trabalho.
No segundo capítulo é apresentada a revisão de literatura no que tange aos principais conceitos que são explorados, como mobilidade, gênero e sustentabilidade. Na sequência, são apresentadas algumas reflexões acerca das intersecções e sobreposições desses aspectos.
No terceiro capítulo são apresentados os métodos que pretendem ser usados no desenvolvimento desta dissertação, a saber: análise das pesquisas Origem Destino realizadas pela Companhia do Metropolitano de São Paulo (Metrô-SP) por meio de um estudo de coorte em pseudo-painel.
No quarto capítulo são apresentadas algumas análises feitas a partir da pesquisa Origem Destino 2007 do Metrô-SP (OD-2007), à luz dos caminhos apontados pela revisão de literatura.
Ao final, são apresentadas algumas conclusões preliminares e também quais são os próximos passos planejados para a continuação deste trabalho.
