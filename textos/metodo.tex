% ---
% Capitulo Métodos
% ---
\chapter{Métodos}\label{chap:metodo}
% META: 10p.

YYYYYYYYYYY

constructo como sendo uma definição mental de uma ideia de pesquisa, estabelecida com base na teoria subjacente e/ou na experiência e na intuição do pesquisador

Análise descritiva de dados 77-2007
dados secundários
ANOVA p/ distância e tempo (homens e mulheres)
chi-quadrado p/ verificar a diferença entre os grupos
SPSS: Analyze > compare means > one way ANOVA
ticar: LSD, TUKEY
Data>split file

Entre os vários modos de transporte, o automóvel é a forma geralmente de maior atratividade. Mas, apesar de seus apelos, um grupo particular de pessoas chama a atenção em relação ao uso diferenciado que fazem do automóvel: as mulheres. Historicamente, estas usam menos o automóvel em relação aos homens. Esta pesquisa de mestrado busca identificar e entender as estratégias utilizadas pelas mulheres em seu acesso em geral mais restrito ao automóvel particular e como este fato afeta seus padrões de atividades e viagens, com o objetivo de formular políticas públicas que estimulem comportamentos similares, menos dependentes  do uso do automóvel.


Strictly the term “sustainable transportation” has little academic meaning
following the World Commission on Environment and Development Commission’s
(Brundtland) Report that initially defined the parameters of sustainable
development and that highlights as its main theme the interconnectivity of
all forms of activity. Here we use the term “sustainable transportation” in
the more popular, journalist sense of reducing the environmental impacts of
transportation.