\newcommand{\partbr}{\index{\emph{Participa.br}}}
\newcommand{\participatorio}{\index{\emph{Participatório}}}
\newcommand{\gabdig}{\index{\emph{Gabinete Digital}}}
\newcommand{\edem}{\index{\emph{e-Democracia}}}
\newcommand{\ptrans}{\index{\emph{Portal da Transparência}}}

\definecolor{lightblue}{rgb}{.80,.85,1}
\definecolor{lightred}{rgb}{1,.80,.85}
\definecolor{lightgreen}{rgb}{.80,1,.85}

% COMANDOS PARA NOTAS DE "TODO"
\newcounter{todocounter}
%To-do que indica necessidade de correção de referência
\newcommand{\corrigeref}[1]{
    \sethlcolor{lightgreen}
    #1\hl{(Corrigir referência)}\todonum[color=lightgreen]{Corrigir Referência}
    \sethlcolor{yellow}
}
%To-do que destaca 'highlight' o texto e adiciona a nota também
\newcommand{\todohl}[2]{
    \hl{#1}\todonum{#2}
}
%To-do da cor vermelha, revisão do texto necessária
\newcommand{\todorevisar}[1]{
    \sethlcolor{lightred}
    \hl{#1}\todonum[color=lightred]{Revisar o trecho}
    \sethlcolor{yellow}
}
%To-do da cor azul, vindo de "Recorte e cole", possivelmente não vai ficar
\newcommand{\todojogado}[1]{#1\todonum[color=blue!40]{Texto vindo de Recorte e Cole}}
%To-do padrão numerado. Contém um parâmetro opcional para referência da localização da nota
\newcommand{\todonum}[2][]{\stepcounter{todocounter}\todo[#1]{\thetodocounter: #2}}

%%%%%%%%%%%%%%%%%%%%%%%%%%%%%%%%%%%%%%%%%%%%%%%%%%%%%%%%%%%%%%%%%%%%%%%%%%%%%%%%%%%%%%%%%%%%%%%
% AMBIENTE E LISTA DE QUADROS
% Novo list of (listings) para QUADROS
\newcommand{\quadroname}{Quadro}
\newcommand{\listofquadrosname}{Lista de quadros}
\newfloat[chapter]{quadro}{loq}{\quadroname}
\newlistof{listofquadros}{loq}{\listofquadrosname}
\newlistentry{quadro}{loq}{0}

% configurações para atender às regras da ABNT
\counterwithout{quadro}{chapter}
\renewcommand{\cftquadroname}{\quadroname\space}
\renewcommand*{\cftquadroaftersnum}{\hfill--\hfill}



% Para centralizar células de tabelas
\newcolumntype{P}[1]{>{\centering\arraybackslash}p{#1}}
\newcolumntype{M}[1]{>{\centering\arraybackslash}m{#1}}

% Para rotacionar os títulos de forma adequada
\newcommand{\turn}[3][10em]{% \turn[<width>]{<angle>}{<stuff>}
  \rlap{\rotatebox[origin=rB]{#2}{\begin{varwidth}[t]{#1}\bfseries#3\end{varwidth}}}%
}

% Comandos de cores de céluas
%%Cor Padrão de cabeçalhos de tabelas
\newcommand{\headerColor}{RoyalBlue!95!black!20}
\newcommand{\headerFontStyle}{\sffamily\bfseries\color{white}}
\newcommand{\headerCell}[1]{
    %\multicolumn{1}{c|}{\cellcolor{\headerColor}\textcolor{white}{\sffamily\bfseries{#1}}}
    %\cellcolor{\headerColor}\textcolor{white}{\sffamily\bfseries{#1}}
    \textbf{#1}
}

\newcommand{\destValCel}[1]{\mbox{\color{Red!99!black!99}#1}}

\newcommand{\headerCenterCell}[1]{
    %\multicolumn{1}{c|}{\cellcolor{\headerColor}\textcolor{white}{\sffamily\bfseries{#1}}}
    %\cellcolor{\headerColor}\textcolor{white}{\sffamily\bfseries{#1}}
    \begin{center}\textbf{#1}\end{center}
}

\newcommand{\headerCenter}[3]{%
    %#1 = Qtde de Células
    %#2 = tamanho da célula
    %#2 = conteúdo
    \multicolumn{#1}{P{#2}}{\textbf{#3}}%
}%

\newcommand{\CellCenter}[3]{%
    %#1 = Qtde de Células
    %#2 = tamanho da célula
    %#2 = conteúdo
    \multicolumn{#1}{P{#2}}{#3}%
}%

\newcommand{\headerCenterM}[3]{%
    %#1 = Qtde de Células
    %#2 = tamanho da célula
    %#2 = conteúdo
  \multicolumn{#1}{|M{#2}}{{\headerCenterCell{\textbf{#3}}}}%
}%

\newcommand{\headerTabCenterCell}[1]{
    \multicolumn{1}{c}{\textbf{#1}}
}
\newcommand{\celAlinhaEsquerda}[1]{
    \multicolumn{1}{l}{#1}
}
\newcommand{\celAlinhaDireita}[1]{
    \multicolumn{1}{r}{#1}
}
\newcommand{\celAlinhaCentro}[1]{
    \multicolumn{1}{c}{#1}
}
\newcommand{\destaqueCel}{
    \cellcolor{\headerColor}
}


%%%%%%%%%%%%%%%%%%%%%%%%%%%%%%%%%%%%%%%%%%%%%%%%%%%%%%%%%%%%%%%%%%%%%%%%%%%%%%%%%%%%%%%%%%%%%%%
% Refazendo comando de capa e folha de rosto
\renewcommand{\imprimircapa}{%
  \begin{capa}%
    \center
    
    \imprimirinstituicao

    \ABNTEXchapterfont\large\imprimirautor

    \vfill
    \ABNTEXchapterfont\bfseries\LARGE\imprimirtitulo
    \vfill

    \large\imprimirlocal

    \large\imprimirdata

    \vspace*{1cm}
  \end{capa}
}

\renewcommand{\folhaderostocontent}{
    \begin{center}

      %\vspace*{1cm}
      \imprimirinstituicao

      {\ABNTEXchapterfont\large\imprimirautor}

      \vspace*{\fill}\vspace*{\fill}
      \begin{center}
      \ABNTEXchapterfont\bfseries\Large\imprimirtitulo
      \end{center}
      \vspace*{\fill}

      \hspace{.45\textwidth}
      \begin{minipage}{.5\textwidth}
      \SingleSpacing
      \imprimirpreambulo
      \end{minipage}%
      \vspace*{\fill}

      {\large\imprimirorientadorRotulo~\imprimirorientador\par}
      \large\imprimircoorientadorRotulo~\imprimircoorientador
      \vspace*{\fill}

      {\large\imprimirlocal}
      \par
      {\large\imprimirdata}
      \vspace*{1cm}

    \end{center}
  }
